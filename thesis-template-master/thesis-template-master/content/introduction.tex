\chapter{Einführung}

In den letzten Jahren stieg die Anzahl der Smart Home Geräte stark. Laut Verified Market Research wird
der Smart Home Markt im Jahr 2019 auf bereits 80 Milliarden Dollar geschätzt. Prognostiziert werden 207.88 
Milliarden Dollar im Jahr 2027. Dies entspricht einer jährlichen Wachstumsrate von 13.52 Prozent.
~\cite{https://www.verifiedmarketresearch.com/product/global-smart-home-market-size-and-forecast-to-2025/}
Zudem ist es so, dass innerhalb von Smart Home Systemen die Anzahl der Geräte ebenfalls stetig steigt. Während 2016
durchschnittlich 5.8 Geräte über ein Smart Home System vernetzt waren, waren es 2018 schon bereits 8.1 Geräte. 
~\cite{https://www.homeandsmart.de/} Es ist zu vermuten, dass sich dieser Trend vorerst so weiterentwickeln wird.
Logischerweise steigt mit der Anzahl der Geräte in einem Smart Home System auch die Anzahl der Updates und somit 
insgesamt die Komplexität von Smart Home Systemen. Das Problem hierbei ist, dass mit steigender Komplexität zum einen automatisch eine höhere
Fehleranfälligkeit bezüglich Updates einhergeht, zum anderen wird es schwieriger Fehler zu beheben,
die durch Updates verursacht werden. Updates im Einzelnen besitzen bereits viele Fehlerquellen, wie zum Beispiel Sicherheitslücken 
oder generell fehlerhafte Implementationen. Dies sind jedoch Aspekte die bereits vom Entwickler weitestgehend vermieden werden.
Viel Problematischer ist das Zusammspiel verschiedener Geräte und den dazugehörigen Updates. Ein an sich fehlerfreies Update
kann bei der Installtion in einem Smart Home System, dennoch große Probleme verursachen. Was ist wenn zum Beispiel ein Update eine Funktion
eines Geräts so verändert, dass ein anderes Gerät nicht mehr darauf zugreifen kann. Durch solche kleine Änderungen
können Abhängigkeiten zwischen Geräten zerstört werden, wodurch ganze Teile eines Smart Home Systems ausfallen
können. Solch ein Ausfall ist für den Nutzer meist ein großer finanzieller Schaden, da Updates oft nicht 
rückgängig gemacht werden können und somit Ersatz beschafft werden muss.
Aus diesen Gründen sind präventive Maßnahmen notwendig,  um die Langlebigkeit von Smart Home Systemen gewährleisten zu können.
Langlebigkeit wird nämlich von den meisten Usern erwartet, wie man an der Resonanz zum Abschalten der HueBridge V1 von Philips sehen kann 
~\cite{Paper [4]}






\section{Ziel der Arbeit}
Ziel dieser Arbeit ist die Implementation und Evaluation eines Algorithmus, welcher Abhaengigkeiten zwischen Geraeten untersucht und basierend auf diesen Abhängigkeiten Updatekonfigurationen findet, die keine Probleme verursachen.
Mit diesem Algorithmus wäre es möglich Updates bereits vor der Installation zu überprüfen und so sicherzugehen, dass sie 
keine Ausfälle verursachen. Problem hierbei ist, wie bereits erwähnt, dass es etliche Smart Home Anbieter mit verschiedener
Software und Geräten gibt. Meist findet man selbst innerhalb eines Smart Home Systems bereits Geräte, die nicht vom selben
Hersteller sind. Dies erschwert die Entwicklungs eines Algorithmus, welcher allgemein an allen Arten von Systemen anwendbar sein soll.
Für die Implementation eines Algorithmus ist es daher notwendig gewisse Dinge vorauszusetzen und zu definieren. Zunächst die Definition eines Smart Homes:

~\cite{Paper [12]}
A smart home has a set of devices that are connected through
a platform. Each device has a certain software version and
a set of available updates. Each software version has a set
of available predefined services. Devices can use services of
other devices which creates dependencies. An update configuration
is one of the finite states of the nondeterministic
finite automaton (NFA) that can be constructed by using the
configurations as states and connecting them using individual
updates as transitions.

Nun benötigt man noch eine Möglichkeit die Informationen über die Services und Updates eines Geräts leicht abfragen zu können.
Eine solche Abfrage müsste in der Realität für jedes Gerät individuell angepasst werden. Dies ist jedoch nicht umsetzbar,
weswegen es nötig ist einen einheitlichen Weg zu finden. Geräte besitzen üblicherweise Metadaten, welche Informationen über 
Firmware Version, Größe usw. enthalten. Diesen Metadaten könnte man zusätzlich Informationen über Services und Updates 
hinzufügen, sodass man einen schnellen Überblick über alle Geräte innerhalb eines Netzwerks erhalten kann.

Der entwickelte Algorithmus wird nicht an realen Smart Home Systemen getestet, weswegen sich diese Frage 
zunächst erübrigt. Es bieten sich zwei Möglichkeiten an den Algorithmus zu testen. Eine Möglichkeit ist eine statische Liste, welche man 
man manuell mit Geräten und deren Updates füllt, oder ein Generator, welcher abhängig von Parametern ein "künstliches" Smart Home
System kreiert. Parameter wären dann zum Beispiel die Anzahl der Geräte oder die Anzahl der Updates, die ein einzelnes Gerät besitzt.
Der Generator bietet sich mehr an, da der Algorithmus nicht nur getestet, sondern auch evaluiert werden soll.
Die Evaluation kann dann an unterschiedlich komplexen Smart Home Systemen durchgeführt werden, indem man die Parameter 
dementsprechend verändert. Parameter wären zum Beispiel die Anzahl der Geräte, die Anzahl der Updates die ein Gerät besitzt oder 
auch die Anzahl der Services. 

 

