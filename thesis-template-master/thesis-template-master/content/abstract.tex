\chapter*{Abstract}

Heutzutage erfreuen sich Smart Home Systeme immer größerer Popularität. Ein Smart Home System
ist ein Netzwerk von mehreren Geräten, meist innerhalb eines Wohnraumes. Wie  
jede andere Technologie auch, unterliegen Smart Homes einem ständigen Wandel. Dabei gibt es einen
wichtigen Punkt zu beachten. Während andere Technologien, wie z.B. Smartphones oft einfach 
nach Jahren aufgrund ihres Alters ausgetauscht werden, wird von Smart Home Systemen eine lange
Lebensdauer erwartet. Die meist komplexen und teuren Installationen sind nicht dazu ausgelegt alle paar Jahre 
ausgetauscht zu werden. Dennoch erwartet man, wie von anderen Technologien auch, dass die Systeme
sich den Gegebenheiten der Zeit anpassen. Neue Funktionen oder oft auch Sicherheitslücken erfordern 
ein ständiges updaten der Systeme oder einzelner Geräte. Mit diesen Updates geht jedoch ein großes Risiko 
einher. Aufgrund der riesen Diversität verschiedener Smart Home Systeme kann vom Hersteller meist nicht
garantiert werden, dass ein Update keine Schäden anrichtet. Lediglich das Update im Einzelnen wird auf 
mögliche Fehlerquellen überprüft, nicht jedoch das Zusammenspiel des Updates mit anderen Geräten.
Dies wäre aufgrund der riesen Diversität an Smart Home Systemen für den Hersteller auch gar nicht umsetzbar.
Dennoch benötig man eine Möglichkeit Updates präventiv zu überprüfen, um so größere Schäden zu meiden und 
die Langlebigkeit von Smart Home Systemen zu gewährleisten.
Diese Abschlussarbeit beschäftigt sich mit der Implementation und Evaluation eines Algrorithmus, welcher 
Abhängigkeiten zwischen Geräten untersucht und basierend auf diesen Abhängigkeiten Updatekonfigurationen 
findet, die keine Probleme verursachen.
Der Algorithmus basiert auf dem Paper "An Algorithm for Dependency-Preserving 
Smart Home Updates" von Peter Zdanking, Matthias Schaffeld, Marian Waltereit, Oskar Carl und Torben Weis. 






\footnote{Wikipedia: \url{https://en.wikipedia.org/wiki/Abstract_(summary)}}

