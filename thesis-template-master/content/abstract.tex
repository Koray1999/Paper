\chapter*{Abstract}


%Smart Home Systeme sind Netzwerke elektronischer Geräte mit dem Ziel die Wohn- und Lebensqualität
%ihrer Nutzer im eigenen Heim zu erhöhen.
%Diese Systeme unterliegen, wie jede andere Technologie auch einem ständigen Wandel. Während viele elektronische 
%Geräte, seien es Smartphones, Tastaturen oder Kopfhörer meist nach einigen Jahren ausgetauscht werden, ist die 
%Erwartungshaltung gegenbüber Smart Home Systemen eine andere. Aufgrund der Anschaffungskosten erwarten Käufer
%eine lange Lebensdauer. Gleichzeitig müssen sich die Systeme den Gegebenheiten der Zeit anpassen. Neue Funktionen oder
%Sicherheitslücken erfordern ein ständiges updaten der einzelnen Geräte in einem System. Mit jedem Update geht ein großes Risiko 
%einher, da Hersteller nicht garantieren können, dass ein Update keine Schäden im System verursacht. Dies liegt daran, 
%dass Updates lediglich für das betroffene Gerät auf 
%mögliche Fehlerquellen überprüft werden, nicht jedoch das Zusammenspiel des Updates mit anderen Geräten. 
%Aufgrund der Diversität an Smart Home Systemen und Geräten wäre dies für den Hersteller gar nicht umsetzbar.
%Daher bedarf es einer Möglichkeit Updates präventiv zu überprüfen, um so größere Schäden zu meiden und 
%die Langlebigkeit von Smart Home Systemen gewährleisten zu können.
%Diese Arbeit beschäftigt sich mit der Implementation und Evaluation eines Algorithmus, welcher 
%Abhängigkeiten zwischen Geräten untersucht und basierend auf diesen Abhängigkeiten Updatekonfigurationen 
%ermittelt, welchen den größten Nutzen für den Nutzer haben.
%Der Algorithmus basiert auf dem Paper "An Algorithm for Dependency-Preserving
%Smart Home Updates" (Zdankin et. al) [1].




Smart Home Systeme sind Netzwerke elektronischer Geräte mit dem Ziel die Wohn- und Lebensqualität
ihrer Nutzer im eigenen Heim zu erhöhen.
Diese Systeme unterliegen, wie jede andere Technologie auch, einem ständigen Wandel. Neue Funktionen oder
Sicherheitslücken erfordern ein ständiges Updaten der einzelnen Geräte in einem System.
Gleichzeitig erwarten Käufer aufgrund der Anschaffungskosten
eine lange Lebensdauer. Mit jedem Update ist jedoch ein Risiko verbunden,
da Hersteller nicht garantieren können, dass ein Update keine Schäden im System verursacht.
Dies liegt darin begründet, dass Updates lediglich für das betroffene Gerät auf 
mögliche Fehlerquellen überprüft werden, nicht jedoch das Zusammenspiel des Updates mit anderen Geräten. 
Aufgrund der Diversität an Smart Home Systemen und Geräten wäre dies für Hersteller nicht umsetzbar.
Daher bedarf es einer Möglichkeit Updates präventiv zu überprüfen, um so größere Schäden zu meiden und 
die Langlebigkeit von Smart Home Systemen gewährleisten zu können.
Diese Arbeit beschäftigt sich mit der Implementation und Evaluation eines Algorithmus, welcher 
Abhängigkeiten zwischen Geräten untersucht und basierend auf diesen Abhängigkeiten Updatekonfigurationen 
ermittelt, welche sicher sind und den größten Nutzen bieten.
Der Algorithmus basiert auf dem Paper "An Algorithm for Dependency-Preserving
Smart Home Updates" (Zdankin et. al) [1].