\chapter*{Abstract}


Smart Home Systeme sind Netzwerke elektronischer Geräte, welche die Funktion haben die Wohn- und Lebensqualität
ihrer Nutzer im eigenen Heim zu erhöhen.
Diese Systeme unterliegen, wie jede andere Technologie auch, einem ständigen Wandel. Während jedoch viele elektronische 
Geräte, seien es Smartphones, Tastaturen oder auch Kopfhörer meist nach einigen wenigen Jahren ausgetauscht werden, ist die 
Erwartungshaltung gegenbüber Smart Home Systemen eine andere. Aufgrund der hohen Anschaffungskosten erwarten Käufer
eine hohe Lebensdauer. Gleichzeitig müssen sich die Systeme den Gegebenheiten der Zeit anpassen. Neue Funktionen oder oft
auch Sicherheitslücken erfordern ein ständiges updaten der einzelnen Geräte in einem System. Mit jedem Update geht jedoch ein großes Risiko 
einher, da Hersteller nicht garantieren können, dass ein Update keine Schäden im System anrichtet. Dies liegt daran, 
dass Updates lediglich für das betroffene Gerät auf 
mögliche Fehlerquellen überprüft werden, nicht jedoch das Zusammenspiel des Updates mit anderen Geräten. 
Aufgrund der Diversität an Smart Home Systemen und Geräten wäre dies für den Hersteller gar nicht umsetzbar.
Daher bedarf es einer Möglichkeit Updates präventiv zu überprüfen, um so größere Schäden zu meiden und 
die Langlebigkeit von Smart Home Systemen zu gewährleisten zu können.
Diese Arbeit beschäftigt sich mit der Implementation und Evaluation eines Algrorithmus, welcher 
Abhängigkeiten zwischen Geräten untersucht und basierend auf diesen Abhängigkeiten Updatekonfigurationen 
ermittelt, welchen den größten Nutzen für den Nutzer haben.
Der Algorithmus basiert auf dem Paper "An Algorithm for Dependency-Preserving 
Smart Home Updates" von Peter Zdanking, Matthias Schaffeld, Marian Waltereit, Oskar Carl und Torben Weis[1]. 






\footnote{Wikipedia: \url{https://en.wikipedia.org/wiki/Abstract_(summary)}}

