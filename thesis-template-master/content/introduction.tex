\chapter{Einführung}

In den letzten Jahren erfreuen sich Smart Home Systeme immer größerer Beliebtheit. Laut Verified Market Research wird
der globale Smart Home Markt im Jahr 2019 auf bereits 80 Milliarden Dollar geschätzt. Prognostiziert werden 207.88 
Milliarden Dollar im Jahr 2027, was einer einer jährlichen Wachstumsrate von 13.52 Prozent entsprechen würde[2].
Diese Zahlen spiegeln den Wunsch der Menschen nach digitaler Vernetzung im eigenen Heim wider, um das eigene Heim
komfortabler und sicherer zu gestalten [3].
Um den Ansprüchen der Nutzer gerecht zu werden, entwickeln Anbieter immer neuere und bessere Techonologien,
wodurch die Systeme immer umfangreicher und komplexer werden. Während 2016
durchschnittlich 5.8 Geräte über ein Smart Home System vernetzt waren, waren es 2018  bereits durchschnittlich 8.1 Geräte [4]. 
Es ist zu vermuten, dass sich dieser Trend vorerst so weiterentwickeln wird und somit Smart Home Systeme immer
größer und komlexer werden.
Wie jede andere Technologie auch müssen Smart Home Systeme regelmäßig geupdatet werden, um zum Beispiel
Sicherheitslücken zu schließen oder neue Funktionalitäten hinzuzufügen. Ein Smart Home System besteht aus vielen verschiedenen
Komponenten, weswegen es nicht möglich ist das System als Ganzes upzudaten. Vielmehr werden einzelne Komponenten eines Systems
unabhängig voneinander aktualisiert. Problem hierbei ist, dass Software Updates potenzielle Fehlerquellen beinhalten können, wie
fehlerhafte Implementationen. Dies sind jedoch Aspekte die bereits vom Entwickler weitestgehend vermieden werden.
Viel Problematischer ist das Zusammspiel verschiedener Geräte und den dazugehörigen Updates. Ein an sich fehlerfreies Update
kann bei der Installtion in einem Smart Home System, trotz "fehlerfreier" Programmierung Probleme verursachen. Mit Problemen
ist in diesem Kontext gemeint, dass ein Update die Kommunikation zwischen zwei Geräten in einem System unterbrechen kann.
Was ist wenn zum Beispiel 
ein Update eine Funktion eines Geräts so verändert, dass ein anderes Gerät nicht mehr darauf zugreifen kann. Durch solche Änderungen
können Abhängigkeiten zwischen Geräten zerstört werden, wodurch ganze Teile eines Smart Home Systems ausfallen
können. Solch ein Ausfall bedeutet für den Nutzer meist einen großen finanziellen Schaden, da Updates oft nicht 
rückgängig gemacht werden können und somit Ersatz beschafft werden muss[1].
Aktueller Stand im Bereich Smart Home ist, dass es keine Möglichkeit gibt Updates vor ihrer Installation zu überprüfen, so dass man
als Nutzer vollkommen auf die vielen Entwickler angewiesen ist. 
Aus diesen Gründen sind präventive Maßnahmen notwendig,  um die Langlebigkeit von Smart Home Systemen gewährleisten zu können.



\section{Ziel der Arbeit}

Eine Steigerung der Lebensqualität ist das Kernziel von Smart Home Systemen. Dies erreicht man zum Beispiel durch die Konfiguration von 
automatisierten Prozessen. Dadurch können Nutzer Zeit, Energie und in gewissen Belangen auch Geld sparen. Laut Umfragen geben 57 Prozent der Nutzer von
Smart Home Systemen in Amerika an jeden Tag durchschnittlich 30 Minuten Zeit einzusparen [5].
Zusätzlich lassen sich nach Berechnungen des Fraunhofer Instituts für Bauphysik  durch Automatisierungen bis zu ca. 40 Prozent Heizkosten 
sparen [6].
Solche Vorteile zeigen, dass sich die anfangs teure Anschaffung eines Smart Home Systemes auf lange Zeit rentieren kann. Sicherlich 
ist dies der Idealfall, aber dennoch sind die genannten Zahlen Indikatoren dafür, dass Smart Home Systeme viele Potenziale bieten. Es hängt am Ende von 
vielen Faktoren ab, ob sich die Anschaffung eines Smart Homes lohnt und objektiv ist dies nicht zu beurteilen. Was jedoch 
objektiv gesagt werden kann, ist dass in den meisten Fällen Nutzer aufgrund der teuren Anschaffungskosten eines Smart Homes davon ausgehen, dass 
ihr System viele Jahre erhalten bleibt. Zum Beispiel gibt es Smart Home Systeme, welche beim Bau eines Hauses direkt mit geplant werden,
sodass man, wie zum Beispiel bei KNX Systemen alle Geräte über ein permanent installiertes Bus System verbinden kann [1].
Selbstverständlich gibt es auch günstigere Varianten, aber dennoch ist die generelle Erwartungshaltung, dass ein funktionierendes System
auch über viele Jahre hinweg weiterhin funktionieren wird.
Daher ist das Ziel dieser Arbeit die Implementation und Evaluation eines Algrorithmus, welcher 
Abhängigkeiten zwischen Geräten untersucht und basierend auf diesen Abhängigkeiten Updatekonfigurationen 
ermittelt, welchen den größten Nutzen für den Nutzer haben.
Mit diesem Algorithmus wäre es für Smart Home Nutzer möglich Updates bereits vor der Installation zu überprüfen, um so sicherzustellen, 
dass sie keine Ausfälle verursachen. Die Anwendung dieses Algorithmus sollte für den Nutzer möglichst einfach sein. So könnte man zum 
Beispiel ein Smartphone nutzen, welches als zentrales Gerät im System dient. Auf diesem zentralen Gerät wäre es dann möglich den 
Algorithmus laufen zu lassen und sich über das Smartphone dann für die passende Updatekonfiguration zu entscheiden. Smartphones
bieten sich besonders an, da sie eine intuitive Bedienung für den Nutzer ermöglichen und im Normalfall jeder Nutzer eines Smart Homes
ein Smartphone besitzt. So umgeht man das Problem, dass es, wie bereits erwähnt, etliche Smart Home Anbieter mit verschiedener
Software und Geräten gibt. 


\section{Aufbau der Arbeit}

Kapitel 2 befasst sich mit aktuellen Technologien im Smart Home Bereich und bildet eine Grundlage für das Verständnis der
restlichen Arbeit. Die Literatur auf der diese Arbeit aufbaut wird vorgestellt und zusätzlich werden die Themen Langlebigkeit
und Dependency Management im Software Bereich untersucht.
Im dritten Kapitel 3 wird das Design und die Implementation des Algorithmus behandelt.
Kapitel 4 umfasst eine Evalutation des Algorithmus hinsichtlich Performance und Designentscheidungen.
Das letzte Kapitel fasst die Ergebnisse der Arbeit zusammen und thematisiert mögliche Verbesserungen der Arbeit.


