\chapter{Einführung}

In den letzten Jahren stieg die Anzahl der Smart Home Geräte stark. Laut Verified Market Research wird
der Smart Home Markt im Jahr 2019 auf bereits 80 Milliarden Dollar geschätzt. Prognostiziert werden 207.88 
Milliarden Dollar im Jahr 2027. Dies entspricht einer jährlichen Wachstumsrate von 13.52 Prozent [1].
Zudem ist es so, dass innerhalb von Smart Home Systemen die Anzahl der Geräte ebenfalls stetig steigt. Während 2016
durchschnittlich 5.8 Geräte über ein Smart Home System vernetzt waren, waren es 2018 schon bereits 8.1 Geräte [2]. 
Es ist zu vermuten, dass sich dieser Trend vorerst so weiterentwickeln wird und somit Smart Home Systeme immer
größer werden. Wie jede andere Technologie müssen Smart Home Systeme regelmäßig geupdatet werden, um zum Beispiel
Sicherheitslücken zu schließen oder neue Funktionalitäten hinzuzufügen. Dies geschieht nicht zentral für das gesamte 
Netzwerk, sondern meist wird jedes Gerät unabhängig von den anderen geupdatet.
 Mit der steigenden Anzahl an Geräten, steigt in einem Smart Home System auch logischerweise die Anzahl der Updates und somit 
insgesamt die Komplexität von Smart Home Systemen. Das Problem hierbei ist, dass mit steigender Komplexität zum einen automatisch eine höhere
Fehleranfälligkeit bezüglich Updates einhergeht, zum anderen wird es schwieriger Fehler zu beheben,
die durch Updates verursacht werden. Updates im Einzelnen besitzen bereits viele Fehlerquellen, wie zum Beispiel Sicherheitslücken 
oder generell fehlerhafte Implementationen. Dies sind jedoch Aspekte die bereits vom Entwickler weitestgehend vermieden werden.
Viel Problematischer ist das Zusammspiel verschiedener Geräte und den dazugehörigen Updates. Ein an sich fehlerfreies Update
kann bei der Installtion in einem Smart Home System, trotz "fehlerfreier" Programmierung große Probleme verursachen. Was ist wenn zum Beispiel 
ein Update eine Funktion eines Geräts so verändert, dass ein anderes Gerät nicht mehr darauf zugreifen kann. Durch solche kleine Änderungen
können Abhängigkeiten zwischen Geräten zerstört werden, wodurch ganze Teile eines Smart Home Systems ausfallen
können. Solch ein Ausfall bedeutet für den Nutzer meist ein großer finanzieller Schaden, da Updates oft nicht 
rückgängig gemacht werden können und somit Ersatz beschafft werden muss.
Aus diesen Gründen sind präventive Maßnahmen notwendig,  um die Langlebigkeit von Smart Home Systemen gewährleisten zu können.
Langlebigkeit wird nämlich von den meisten Usern erwartet, wie man an der Resonanz zum Abschalten der HueBridge V1 von Philips sehen kann 
[3].






\section{Ziel der Arbeit}


Eine Steigerung der Lebensqualität ist das Kernziel von Smart Home Systemen. Dies erreicht man zum Beispiel durch Konfiguration von 
automatisierten Prozessen. Dadurch können Nutzer nämlich Zeit, Energie und in gewissen Belangen auch Geld sparen. Laut Umfragen geben 57 Prozent der Nutzer von
Smart Home Systemen in Amerika an jeden Tag durchschnittlich 30 Minuten Zeit zu sparen (https://policyadvice.net/insurance/insights/smart-home-statistics/).
Zusätzlich lassen sich nach Berechnungen des Fraunhofer Instituts für Bauphysik  durch Automatisierungen bis zu ca. 40 Prozent Heizkosten 
sparen. (https://www.ibp.fraunhofer.de/content/dam/ibp/ibp-neu/de/dokumente/sonderdrucke/bauphysik-gertis/6-einsparpotenziale-intelligente-heizungsregelung.pdf).
Solche Vorteile zeigen, dass sich die anfangs teure Anschaffung eines Smart Home Systemes auf lange Zeit sogar rentieren kann. Sicherlich 
ist dies der Idealfall, aber dennoch zeigen diese Zahlen, dass Smart Home Systeme viele Potenziale beinhalten. Es hängt jedoch von 
vielen Faktoren ab, ob sich die Anschaffung eines Smart Homes lohnt und objektiv ist dies eigentlich gar nicht zu beurteilen. Was man jedoch 
objektiv sagen kann, ist dass Nutzer aufgrund der teuren Anschaffungskosten eines Smart Homes mit Sicherheit davon ausgehen, dass 
ihr System viele Jahre erhalten bleibt. Zum Beispiel gibt es Smart Home Systeme, welche beim Bau eines Hauses direkt mit geplant werden,
sodass man, wie zum Beispiel bei KNX alle Geräte über ein permanent installiertes Bus System verbinden kann. (Paper).
Natürlich gibt es auch günstigere Varianten, aber dennoch erwartet man als Nutzer generell, dass wenn man ein funktionierendes System hat, dieses
auch über viele Jahre hinweg weiter funktionieren wird.
Daher is das Ziel dieser Arbeit die Implementation und Evaluation eines Algorithmus, welcher Abhängigkeiten zwischen Geräten 
untersucht und basierend auf diesen Abhängigkeiten Updatekonfigurationen findet, die keine Probleme verursachen.
Mit diesem Algorithmus wäre es für Smart Home Nutzer möglich Updates bereits vor der Installation zu überprüfen, um so sicherzugehen, 
dass sie keine Ausfälle verursachen. Die Anwendung dieses Algorithmus sollte für den Nutzer möglichst einfach sein. So könnte man zum 
Beispiel ein Smartphone nutzen, welches als zentrales Gerät im System dient. Auf diesem zentralen Gerät wäre es dann möglich den 
Algorithmus laufen zu lassen und sich über das Smartphone dann für die passende Updatekonfiguration zu entscheiden. Smartphones
bieten sich besonders an, da sie eine intuitive Bedienung für den Nutzer ermöglichen und im Normalfall jeder Nutzer eines Smart Homes
ein Smartphone besitzt. So umgeht man das Problem, dass es, wie bereits erwähnt, etliche Smart Home Anbieter mit verschiedener
Software und Geräten gibt. 




 

