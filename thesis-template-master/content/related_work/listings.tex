\section{Code Listings}

If there is some interesting code you would like to show
in order to ease the understanding of the text,
you can just include it using the \verb+lstlisting+ environment.
Have a look at the source of this page to see how this is included:

\begin{lstlisting}[language=Go]
x := from(42);
\end{lstlisting}

You could also put the code into an external file
and include it in this document using the \verb+lstinputlisting+ command:

\lstinputlisting[language=Go,numbers=left]{listings/example.go}

Be careful not to include large files as it hampers readability.
If there is a short excerpt from a large file you would like to show,
you can also extract an explicit range of lines from it without the need to modify the source file.
This next listing only shows the conditional from the previous code:

\lstinputlisting[language=Go,numbers=left,firstline=2,lastline=5,firstnumber=2]{listings/example.go}

To use more advanced syntax highlighting
have a look at the available options of the \emph{listings} package
or use the \emph{minted} package\footnote{\url{https://texdoc.net/texmf-dist/doc/latex/minted/minted.pdf}},
which has more extensive language support and additional themes.
Both can be configured in the main file.
