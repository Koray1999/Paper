\chapter{Evaluation}\label{ch:eval}

Im folgenden Kapitel wird die Implementation hinsichtlich ihrer Performance untersucht, wobei der Algorithmus an unterschiedlichen Systemen
getestet wird, die mit Hilfe des Generators erstellt werden. 

\section{Theoretische Performance}
Ein Ziel dieser Evaluation ist es herauszufinden, ob der Algorithmus an aktuellen Smart Home Systemen eine akzeptable Laufzeit vorweist.
Bereits ohne Messungen vorgenommen zu haben, können bei genauerer Betrachtung des Algorithmus Vermutungen zur Komplexität einzelner Schritte aufgestellt werden.
Zum Berechnen des Updatekonfigurationsgraphen muss das kartesische Produkt zwischen allen Geräten und den dazugehörigen Updates gebildet werden. 
Sei ein System mit n Geräten gegeben und pro Gerät existieren s Updates, dann entstehen beim Erstellen des kartesischen Produkts  \( s^n\) mögliche Konfigurationen.
Zur Vereinfachung wird davon ausgegangen, dass alle Geräte in einem System die gleiche Anzahl an Updates besitzten. Die Komplexität dieses Schrittes wäre also
vereinfacht O( \( s^n\)). Für ein "reales"  Smart Home System, in dem die Geräte unterschiedlich viele Updates besitzen, läge die Komplexität bei
O($\vert$Gerät1$\vert$*$\vert$Gerät2$\vert$*$\vert$Gerät3$\vert$*...).
Aufgrund dieses exponentiellen Wachstums ist zu vermuten, dass der Rechenaufwand mit jedem weiteren Gerät und jedem Update stark ansteigt.
Bei einem Smart Home System mit 10 Geräten und jeweils 5 Updates pro Gerät entstehen ohne weitere Optimierungen bereits \( 5^(10) = 9765625\)
verschiedene Konfigurationen. Sollte der Trend sich fortsetzen, dass die durchschnittliche Anzahl an Geräten pro Smart Home System weiterhin ansteigt, könnte
es notwendig sein zusätzliche Optimierungsschritte zu entwickeln, um so den Rechenaufwand zu verringern.
Im Idealfall ergeben die Messungen, dass der Rechenaufwand des Algorithmus so gering ist, dass die Laufzeit auch bei großen Systemen
mit zum Beispiel 30 Geräten vertretbar ist. So kann die Zukunftstauglichkeit des Alogrithmus bei immer größer werdenden Smart Home Systemen
gewährleistet werden.
Ein weiterer rechenintensiver Schritt ist das Löschen aller ungültigen Updatekonfigurationen. Zu diesem Zweck muss für jede potenzielle Updatekonfiguration
jeweils eine Liste mit allen angebotenen Dienstleistungen erstellt werden, die dann für weitere Berechnungen des Algorithmus genutzt werden. 
Für die aus dem vorigen Beispiel \( 5^(10) = 9765625\) Konfigurationen müssten
also 9765625 Listen erstellt werden, die alle einzeln durchgegangen werden, um so ungültige Updatekonfigurationen zu finden. Auch für diesen Teilschritt 
ist somit eine hoher Rechenaufwand zu erwarten. Während andere
Berechnungsschritte ebenfalls zu Laufzeit des Algorithmus beitragen, ist davon auszugehen, dass sie um einen großen Faktor weniger
Rechenintensiv sind und dementsprechend weniger ins Gewicht fallen. Die in diesem Abschnitt aufgestellten Hypothesen werden in den folgenden Abschnitten
überprüft. 

\section{Messungen}
In diesem Kapitel werden mit Hilfe eines Generators verschiedene Smart Home Systeme kreiert, an denen anschließend die Laufzeit des Algorithmus gemessen wird. 
Die Laufzeit des Algorithmus wird stark von der Größe der erstellten Smart Home Systemen abhängig sein. Es stehen folgende Parameter zur Verfügung, um
die Größe eines Smart Home Systems festzulegen: 
\begin{itemize}
\item \#Geräte: Legt die Anzahl der Geräte im System fest
\item \#DienstleistungenProGerät: Legt die Anzahl der Dienstleistungen fest, die ein Gerät anbietet.
\item \#UpdatesProGerät: Legt die Anzahl der Updates pro Gerät fest (Aus Implementationsgründen besitzen alle Geräte die gleiche Anzahl an möglichen Updates)
\item \#Abhängigkeiten: Legt die Anzahl der Abhängigkeiten im System fest.
\end{itemize}
Als erstes soll die Frage beantwortet werden, ob der Algorithmus an Systemen mit aktuell durchschnittlicher Größe anwendbar ist.
Zu diesem Zweck wird mit Hilfe des Generators ein durchschnittliches Smart Home System generiert.
Zunächst muss jedoch geklärt werden wie ein solches Durchschnittssystem aussieht. Laut Statista[8] waren
in amerikanischen Haushälten im Jahr 2020 durchschnittlich 10 Geräte miteinander verbunden. Die durchschnittliche Anzahl an Updates pro Gerät zu
ermitteln, ohne empirische Daten vorliegen zu haben, stellt sich als schwierig dar. Um diesen Parameter nicht willkürlich zu würfeln, wird sich an der Anzahl
der Updates von Apples Iphones orientiert. Seit 2013 erhalten Iphones durchschnittlich 10 Updates pro Jahr[14]. Dabei muss beachtet werden, dass sich die
Upates über das Jahr verteilen und somit kein Gerät von einen auf den anderen Tag 10 neue Updates erhält. Dennoch kann man nicht davon ausgehen,
dass Nutzer täglich überprüfen, ob ein neues Update zur Verfügung steht, so dass sie sich anhäufen. Um täglich manuelle Konfiguration zu vermeiden,
wird davon ausgegangen, dass ein Nutzer regelmäßig alle 3 Monate, also ingesamt 4 mal im Jahr, seine Geräte updatet. So sollte ein Gerät durchschnittlich
zwischen 2 und 3 (10 Updates pro Jahr / 4 mal updaten pro Jahr = 2.5)  möglichen Updates besitzen. 
Nun stellt sich noch die Frage wie viele Dienstleistungen
ein Gerät durchschnittlich anbietet und wie viele Abhängigkeiten zwischen den Geräten herrschen. Auch dies ist ohne empirische Daten schwierig zu beurteilen.
Bei Smart Home Geräten handelt es sich meist um simple Geräte, die nur wenigen Zwecken dienen. So kann zum Beispiel ein Thermostat nur die Temperatur messen,
eine Heizung heizen, eine Sicherheitskamera aufnehmen und vielleicht einen Alarm auslösen. Die meisten Geräte bieten also nur wenige Dienstleistungen. Diese
Annahme beruht jedoch auf eigenen Erfahrungen und könnte somit zu gering geschätzt sein. Daher wird für das Erstellen der Geräte davon ausgegangen, dass sie durchschnittlich
5 Dienstleistungen anbieten, um so die Größe eines Systems nicht zu unterschätzen. Aufgrund der geringen Anzahl an Dienstleistungen pro Gerät gehen wir ebenfalls 
davon aus, dass die Anzahl der Abhängigkeiten, die ein Gerät besitzt ebenfalls gering ist. Viele Geräte funktionieren auch vollkommen selbstständig besitzen und beseitzen keinerlei
Abhängigkeit. Für die Anzahl der Abhängigkeiten wird daher festgelegt, dass ein Gerät durchschnittlich 2 Abhängigkeiten besitzt.
\begin{figure}[h]
\begin{center}
\includegraphics[width=14cm,height=9cm]{"Durchschnitt"}
\caption{Laufzeitmessungen an einem durchschnittlichen System}
\label{fig:Prob1:MEA}
\end{center}
\end{figure}
\FloatBarrier
Anhand der Abbildung 4.1 ist zu erkennen, dass die Berechnungen im Bereich von Milisekunden durchlaufen sind. Die durschnittliche Laufzeit des Algorithmus liegt
bei 79 Millisekunden mit Ausreißern bei 350 und 21 Millisekunden. In einem Buch zum Thema Usability Engineering hat Jakob Nielsen verschiedene Grenzwerte zur
Reaktionszeit von Webseiten festgelegt. Diese Werte können als Indikator genutzt werden, um die Laufzeit des Algorithmus zu beurteilen. Sie können 
nicht 1 zu 1 auf die Laufzeit des Algorithmus übertragen werden, da der Alogrithmus in einem anderen Kontext als eine Webseiten zu betrachten ist.
Während Webseiten täglich von Nutzern besucht werden, wird der Algorithmus nur einige male im Jahr verwendet werden. Dementsprechend ist davon
auszugehen, dass die Laufzeiten für Nutzer zumutbar sind, solange sie unterhalb den folgenden Grenzen liegen:
\begin{itemize}
\item 0.1 Sekunden: Prozesse unter 0.1 Sekunden nimmt ein Nutzer als ohne Verzögerung wahr.
\item 1.0 Sekunden: Der Nutzer nimmt eine kurze Verzögerung wahr, die meist als nicht störend interpretiert wird.
\item 10 Sekunden: Für bis zu 10 Sekunden wartet ein Nutzer durchschnittlich auf eine Reaktionen. Verzögert sich eine Reaktion um mehr als 10 Sekunden wechselt ein Nutzer
zu anderen Dingen bis eine Reaktion erscheint.
\end{itemize}

Des Weiteren fällt auf, dass der Algorithmus bei gleich bleibenden Parametern starke Schwankungen aufweist. Dies liegt daran, dass trotz gleich
bleibender Parameter die Komplexität eines Smart Homes variieren kann. Überträgt man dies auf die reale Welt ist dies einfacher zu verstehen, da
es logisch erscheint, dass zwei Systeme trotz gleicher Größe nicht automatisch gleich komplex sind. In einem System können viele
Updates dominiert sein, während dies im anderen System nicht der Fall ist. Dies führt dazu, dass die Optimierungen an beiden Systemen unterschiedlich erfolgreich
ausfallen, so dass in einem System mehr Berechnungen durchgeführt werden müssen.

\newpage
\subsection{Laufzeit der Teilschritte}
In diesem Abschnitt werden die Lauftzeiten der einzelnen Teilschritte untersucht, um so Problemstellen in der Implementation ausfindig zu machen. Abbildung 4.2 zeigt die
Laufzeit der einzelnen Schritte, die in Kapitel 3 beschrieben wurden. Die Messungen wurden wie im vorigen Kapitel an einem durchschnitllichen Smart Home System durchgeführt.
Die genauen Parameter sind der Abbildung zu entnehmen.
Es ist zu erkennen, dass der Schritt "deleteBreakingConigurations" den Großteil der Laufzeit ausmacht. Die Laufzeit aller anderen Schritte summiert sich auf nur 34 Prozent
der gesamten Laufzeit aus. Die Hypothesen aus Abschnitt 4.1 haben sich teilweise bewahrheitet. Das Erstellen des Updatekonfigurationsgraphen nimmt nicht so viel Zeit
in Anspruch wie angenommen.
Um so mehr stimmt, dass das Löschen der ungültigen Konfigurationen sehr rechenintensiv im Vergleich zu den anderen Schritten ist. Möchte man den Alogrithmus optimieren,
dann sollte diesem Schritt besondere Aufmerksamkeit geschenkt werden. Die anderen Teilschritte fallen wie erwartet nicht nicht ins Gewicht und können dementsprechend vernachlässigt
werden.
\begin{figure}[h]
\begin{center}
\includegraphics[width=14cm,height=10cm]{"Teilschritte"}
\caption{Laufzeit einzelner Teilschritte}
\label{fig:Prob1:MEA}
\end{center}
\end{figure}

\subsection{Einfluss einzelner Parameter}
Im weiteren Verlauf werden die Parameter des Smart Home Generators verändert und die damit verbundenen Laufzeitschwankungen gemessen. So kann ermittelt werden
welchen Einfluss die einzelnen Parameter auf die Laufzeit haben. Der erste Parameter, der unabhängig von den anderen untersucht wird,
ist die Anzahl der Geräte in einem Smart Home System. 
Dafür wird die die Entwicklung der Laufzeit bei steigender Anzahl an Geräten im System gemessen. Um den Einfluss der anderen Parameter möglichst gering zu halten,
sind sie, wie in Abbildung 4.3 zu sehen, eingestellt.
In der Abbildung ist zu erkennen, das mit steigender Anzahl an Geräten die Laufzeit des
Algorithmus ansteigt. Im Bereich von 1-8 Geräten hat das Hinzufügen eines Geräte eine durchschnittliche Laufzeitsteigerung von circa 15 Prozent zur Folge. Ab 9-13 Geräten
ergeben die Messungen beim Hinzufügen eines Geräts eine Laufzeitsteigerung von circa 30 Prozent. Ab 14 Geräten erhöht sich sich die Laufzeit mit jedem hinzugefügten
Gerät um jeweils circa 60 Prozent. Die Ergebnisse zeigen, dass das Hinzufügen eines Geräts bei kleinen Systemen (unter 13 Geräten) geringe Laufzeiterhöhungen zur Folge haben.
Je größer ein System ist, desto stärker erhöht sich die sich die Laufzeit mit jedem zusätzlich hinzugefügtem Gerät. Die Messungen wurden für bis zu 20 Geräte in einem
System durchgeführt. Aufgrund der Kurve des Graphen ist zu erwarten, dass der Graph bei noch größeren Systemen immer stärker steigen würde.
\begin{figure}[h]
\begin{center}
\includegraphics[width=14cm,height=8.5cm]{"Steigende Geräteanzahl"}
\caption{Laufzeit beim Erhöhen der Anzahl der Geräte im System}
\label{fig:Prob1:MEA}
\end{center}
\end{figure}

\FloatBarrier
In Abbildung 4.3 handelt es sich um theoretische Ergebnisse, um den unabhängigen Einfluss des Steigerns eines einzelnen Parameters zu messen. In einem realistischem Smart Home System
ist davon auszugehen, dass mit steigender Anzahl an Geräten auch die Anzahl
der Abhängigkeiten im System steigt. Außerderm liegt die Anzahl der Updates pro Gerät in einem realistischem  Smart Home System, wie bereits erwähnt zwischen 2 und 3. Um also
praxisnähere Ergebnisse zu ermitteln, zeigen die Messungen in Abbildung 4.4 die Laufzeit des Algorithmus an realistischeren Systemen.
Es ist zu erkennen, dass das Erhöhen der Anzahl der Geräte einen Ähnlichen Effekt hat wie zuvor . Der Unterschied besteht darin, dass
eine starke Laufzeitsteigerung bereits bei weniger Geräten beginnt und noch stärker ansteigt. Aufgrund des starken Anstiegs war es nicht möglich die Anzahl der Geräte wie in Abbildung
4.3 auf bis zu 20 zu steigern. 
Die durchschnittliche Laufzeiterhöhung pro hinzugefügtem liegt bei 1-6 Geräten bei circa 30 Prozent (doppelt so viel wie in den Messungen zuvor).
Ab 7-13 Geräten liegt die Steigerung bei circa 80 Prozent und bei der Erhöhung von 13 auf 14
Geräten hat sich die Laufzeit um 440 Prozent erhöht. Somit kann man zusammenfassen, dass die Laufzeit des Algorithmus bei realistischen Systemen sehr stark ansteigt, wenn
die Anzahl der Geräte in einem System steigt.   
\begin{figure}[h]
\begin{center}
\includegraphics[width=14cm,height=9cm]{"Steigende Geräteanzahl real"}
\caption{Laufzeit beim Erhöhen der Anzahl der Geräte in einem realistischem System}
\label{fig:Prob1:MEA}
\end{center}
\end{figure}

\newpage
Nun folgen weitere Messungen an Systemen bei denen die Anzahl der Updates pro Gerät erhöht werden. Es wird wieder versucht den Einfluss der anderen Parameter auf
die Laufzeit möglichst gering zu halten. Die genauen Einstellungen sind der Abbildung 4.5 zu entnehmen.
Die Abbildung zeigt, dass bei der Verdopplung der Updates pro Gerät die Laufzeit sich immer um durchschnittlich 60 Prozent erhöht. Runtergerechent bedeutet dies, dass jedes Update
die Laufzeit um durchschnittlich 6 Prozent erhöht. Es handelt sich wieder um ein exponentielles Wachstum, was im Vergleich zu Abbildung 4.3 jedoch langsamer ansteigt. Die Messungen
zeigen insgesamt, dass das Verdoppeln der Geräte einen stärkeren Einfluss auf die Laufzeit hat, als das Verdoppeln der Updates im System. 
Dies könnte daran liegen das beim Bilden des kartesischen Produkts, wie in Kapitel 4.1 erwähnt, die Anzahl der Geräte im Exponenten steht und die Anzahl der Updates in der Basis.
Ein Gerät hinzuzufügen lässt den Updatekonfigurationsgraphen dementsprechend schneller wachsen, als das Hinzufügen eines Updates.
\FloatBarrier
\begin{figure}[h]
\begin{center}
\includegraphics[width=14cm,height=9cm]{"Steigende Updatezahl"}
\caption{Laufzeit beim Erhöhen der Anzahl der Updates pro Gerät im System}
\label{fig:Prob1:MEA}
\end{center}
\end{figure}
\newpage
In Abbildung 4.5 wurden die Messungen wieder an einem unrealistischen System vorgenommen, um den Einfluss des Steigerns eines einzelnen Parameters zu messen. 
Also werden Messungen an einem realen System durchgeführt. Aufgrund der Implementation des Generators besitzen alle Geräte die gleiche Anzahl
an Updates. Dementsprechend ist es nicht möglich einem System mit 10 Geräten 1 Update hinzuzufügen. Für zukünftige Messungen wäre es von Vorteil
den Generator entsprechend zu ändern, so dass dies möglich ist. Es ist also nicht möglich die Anzahl der Updates um jeweils 1 zu erhöhen, sondern für
alle Geräte wird jeweils 1 Update hinzugefügt. Dementsprechend ist zu sehen, dass die Laufzeit an einem System mit 10 Geräten extrem ansteigt im Gegensatz
zu einem System mit 5 Geräten. Bei beiden Konfiguration ist ein exponentielles Wachstum zu erkennen.

\begin{figure}[h]
\begin{center}
\includegraphics[width=14cm,height=9cm]{"Steigende Updatezahl real"}
\caption{Laufzeit beim Erhöhen der Anzahl der Updates pro Gerät in einem realistischen System}
\label{fig:Prob1:MEA}
\end{center}
\end{figure}

\newpage
\subsection{Einfluss der Parameter im Zusammenspiel}
Als Letztes wird gemessen welchen Einfluss die beiden Parameter \#Geräte und \#UpdateProGerät auf die Laufzeit haben, wenn man sie gleichzeitig erhöht.
In Abbildung 4.7 zeigen die Messungen, dass die Laufzeit ab 7 Geräten mit je 7 Updates pro Gerät extrem ansteigt. Die Algorithmus weist als auch hier
ein sehr starkes wachstum ab einer bestimmten Grenze vor.
\begin{figure}[h]
\begin{center}
\includegraphics[width=14cm,height=9cm]{"Beides"}
\caption{Caption}
\label{fig:Prob1:MEA}
\end{center}
\end{figure}

\newpage
\section{PC-Spezifikationen}
Die Laufzeit des  Alogrithmus ist von vielen Faktoren abhängig. Um die Ergebnisse besser beurteilen zu können folgt eine Liste der verwendeteten
PC Konfiguration:
\begin{itemize}
\item Betriebssystem: Windows 10 Home 
\item CPU: AMD Ryzen 3 3200g (4 Kerne, Taktfrequenz 3.6 GHz) 
\item RAM: 16 GB DDR4 3000 MHz
\item Speichermedium: SANDISK Ultra 3D NVMe SSD (Lesen: 2400 MB/s, Schreiben: 1750 MB/s)
\end{itemize}

Moderne Smartphone besitzen leistungsstarke Prozessoren. Das meist verkaufte Smartphone im Jahr 2021 ist das Apple Iphone 12 mit einem[15]
A14 Apple Bionic Prozessor und 4 GB LPDDR4X (4266 MHz) RAM. Der Prozessor besitzt 2 Hochleistungskerne mit Taktfrequenzen bis zu 3 GHz und
4 weiteren Kernen mit Taktfrequenzen bis zu 1,82 GHz [16]. Die Komponenten des Iphone 12 sind nicht wesentlich Leistungsschwacher als die des
verwendeteten Computers. Der Algorithmus sollte dementsprechend auf einem modernen Smartphone ebenfalls zügig durchlaufen sein. Verbesserungen
der Implementation bezüglich Nebenläufigkeit würden die Performance zusätzlich steigern, so dass der Algorithmus auch auf leistungsschwächeren
Smartphones funktionieren sollte. 



\section{Ergebnisse}
Die Evaluationsschritte zeigen, dass der Algorithmus zum aktuellen Stand der Smart Home Systeme für viele Menschen ein wichtiges Produkt sein könnte.
Nutzern wäre es möglich mit einer simplen Anwendung auf dem Handy die Updates eines Smart Home System zu steuern.
Die Ergebnisse sind jedoch mit Vorsicht zu betrachten, da die Eintstellungen der Parameter auf vielen Annahmen beruhen für die es noch
keine ausreichende Forschungsgrundlage gibt. Für zukünftige Messungen wäre es von Vorteil empirische Daten über Smart Home Systeme
zu sammeln, um so mit Hilfe des Generators möglichst realitätsnahe Systeme zu kreieren.
Die Evaluation hat auch gezeigt, dass die Zukunftstauglichkeit des Algorithmus nicht garantiert ist und weitere Optimierungen vermutlich
notwendig sein werden, um vertretbare Laufzeiten bei immer größer werdenen Smart Home Systemen zu erreichen. Durch die modulare
Implementierung sollte es möglich Änderungen am Alogrithmus vorzunehmen und ihn so zu verbessern.
Ein potenzieller Ansatz den man verfolgen könnte ist die Reihenfolge des Alogrithmus zu ändern. So könnte man im ersten Schritt nach Subnetzwerken
suchen und den Rest des Algorithmus für jedes Subnetzwerk unabhängig von einander laufen lassen. Dadurch müsste der Algorthmus öfter ausgeführt werden,
jedoch ist zu vermuten, dass  die Ausführungszeit auf vielen kleinen Subnetzwerken geringer sein wird, als auf einem großen System.













  





%Da das Erstellen des kartesischen Produkts die Lauftzeit \( s^n\) hat, dann sollten die folgenden Messungen zeigen, dass die Laufzeit langsamer steigen wird. Erhöht man
%die Anzahl der Geräte, erhöht man die den Exponenten, bei der Anzahl der Updates erhöht man die Basis.
%In Abbildung ? sieht man, dass die Annahmen richtig waren und der Graph langsamer steigt.
%Die Ergebnisse scheinen zu zeigen, dass der Algorithmus auf einem Smartphone blablabla
%Verwendetes Skript: %for ($($i = 0;$j = 0); $i -lt 10; $i++){}





