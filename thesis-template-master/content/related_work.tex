\chapter{Related Work}\label{ch:related_work}


In diesem Kapitel werden Smart Home Systeme genaue beleuchtet und  das Thema Dependency Management aufgegriffen, wobei bereits 
bestehende Ansätze/Lösungen diesbezüglich genauer betrachtet. Zusätzlich wird der aktuelle Stand des Dependency Managements im 
Bereich Smart Home untersucht.

\section{Smart Homes}
~\cite{Paper [12]}

Im vorherigen Kapitel wurde bereits erwähnt, dass es viele Arten von Smart Home Systemen gibt. Im Rahmen dieser Arbeit
ist es jedoch nicht notwendig die verschiedenen Smart Home Architekturen genauer zu beleuchten. Dennoch schadet es 
nicht sich einen groben Überblick über die aktuellen Technologien zu verschaffen. Grob kann man die
verschiedenen Architekturen darin unterscheiden, ob sie eine Verknüpfung zu einer Cloud besitzen. Cloud gebundene
Architekturen bieten dem Nutzer meist eine einfachere Installation und Wartung, indem zum 
Beispiel Updates automatisch installiert werden.Gleichzeitig bringen sie aber auch potenzielle Gefahren mit sich. Die 
Abhängigkeit von einer Cloud macht Smart Home Systeme sehr verwundbar. Wird zum Beispiel ein externer Service, wie zum
Beispiel die Cloud ausgeschaltet, kann es zu starken Ausfällen im System führen. [5](Towards Longevity of Smart Homes).
Außerdem sind Geräte, die mit einer Cloud verbunden sind anfälliger gegenüber Angriffe von Außenstehen, die sich Zugriff auf
das System beschaffen wollen.
Nicht Cloud-gebundene Architekturen haben diese Nachteile nicht. Sie sind jedoch meist schwieriger zu implementieren, da 
sie, wie im Falle von OpenHab, viel manuelle Konfiguration benötigen. Sie bieten dadurch jedoch auch mehr Sicherheit, da keine 
starke Abhängigkeit zu anderen Dienstleistern nötig ist
Natürlich könnte man Smart Home Systeme noch weiter spezifizieren, indem man einzelne Komponenten genauer beleuchet und 
verschiedene Smart Home Anbieter miteinander vergleicht, jedoch ist dies, wie bereits erwähnt, im Rahmen dieser 
Arbeit nicht notwendig. Der zu entwickelnde Algorithmus soll nämlich später theoretisch Platformunabhängig anwendbar sein.
Dafür müssen logischerweise einige Aspekte vorausgesetzt werden. Zum Beispiel eine genaue Definition eines Smart Homes.
Unter dem Begriff Smart Home stellen sich nämlich vermutlich verschiedene Personen verschiedene Dinge vor.
Um dies zu vermeiden wird auf folgende Definition zurückgegriffen:

[4]
\textit{A smart home has a set of devices that are connected through
a platform. Each device has a certain software version and
a set of available updates. Each software version has a set
of available predefined services. Devices can use services of
other devices which creates dependencies. An update configuration
is one of the finite states of the nondeterministic
finite automaton (NFA) that can be constructed by using the
configurations as states and connecting them using individual
updates as transitions.}
Wie man sieht ist diese Definition sehr generell gehalten. Auf Smart Home Architektur spezifische Aspekte wird nicht eingegangen,
was im Rahmen dieser Arbeit völlig ausreichend ist.  




Meist findet man selbst innerhalb eines Smart Home Systems bereits Geräte, die nicht vom selben
Hersteller sind. Dies erschwert die Entwicklungs eines Algorithmus, welcher allgemein an allen Arten von Systemen anwendbar 
sein soll. Für die Implementation eines Algorithmus ist es daher notwendig gewisse Dinge vorauszusetzen und zu definieren.
Zunächst die Definition eines Smart Homes:






\section{Dependency Managment}

Unter Dependency Management versteht man im Allgemeinen das Strukturieren von Abhängigkeiten verschiedener 
Systeme/Programme. Es ist oft so, dass Programme nur in Abhängigkeit von anderen Programmen starten können.
Dies stellt auch kein Problem dar, solange die Abhängigkeiten in ihrer Anzahl überschaubar bleiben. 
Realität ist jedoch, dass Programme meist von mehreren Programmen abhängig sind und diese
Prorgramme wiederum abhängig von anderen Programmen sind. Bei zu vielen Abhängigkeiten verliert man 
als Entwickler und auch als Nutzer schnell den Überblick. Dieses Wirrwarr an Abhängigkeiten nennt man auch "Dependency Hell".
Gemeint ist damit, dass es ab einem bestimmten Punkt unmöglich wird den Überblick über alle Abhängigkeiten zu behalten.
Je mehr Abhängigkeiten in einem System vorhanden sind, desto komplexer wird es, wodurch jede Änderung oder 
Erweiterung einen riesen Aufwand mit sich zieht. 

In der Dependency Hell gibt es eine Zusammenfassung der am häufigsten auftretenden Probleme:
\begin{itemize}
 \item Long Chain of Dependencies: Ein System A ist abhängig von einem System B. Das System B wiederum ist abhängig von einem System C.
Möchte man nun System A nutzen benötigt man zusätzlich System B und damit auch System C. Solche lange Ketten können Konflikte verursachen.
(Siehe Conflicting Dependencies)
\item Conflicting Dependencies: 
\item Conflicting Dependencies: 
\item Conflicting Dependencies: 
\item Conflicting Dependencies: 
\end{itemize}

Es existieren bereits viele Lösungen, um gegen diese Probleme vorzugehen. Dazu gehört:
\begin{itemize}
\item Version numbering: 
\item Version numbering: 
\item Version numbering: 
\item Version numbering: 
\item Version numbering: 
\end{itemize}

Bei Smart Home Systemen ist es jedoch so, dass in diesem Bereich Dependency Management völliges Neuland ist.

Dabei werden Smart Home Systeme als 
Luxusgüter angesehen und aufgrund der hohen Preise erwartet der Verbraucher auch lange Lebensspannen der Geräte.
Dies kann aber ohne vernünftiges Dependency Management nicht gewährleistet werden. Natürlich gibt es auch andere Aspekte wie z.B. 
discontinued services oder security issues, die die Lebenserwartung von Smart Home Systemen verringern. Dennoch sollte 
dabei Dependency Management nicht untergehen. Dies sieht man am Beispiel des Logitech Harmony Hub's. Logitech hatte eine 
neue Firmware Version für das Hub veröffentlicht, welche angeblich nur Sicherheitslücken schließen und Bugs fixen sollte.
Nachdem Update kam es jedoch vermehrt zu Ausfällen zwischen dem Hub und third-party Geräten. Den Nutzern waren die Hände 
gebunden, sie konnten nur darauf hoffen, dass Logitech die Probleme wieder behebt. Dies hat Logitech in diesem Fall auch getan,
dennoch zeigt dieses kleine Beispiel, wie leichtsinnig Updates von Nutzern installiert werden. Eine andere Möglichkeit besitzt man auch als 
Nutzer eines Smart Homes nicht wirklich, da es, wie eingangs erwähnt, zur Zeit keine Möglichkeiten gibt Updates vor der Installation zu 
überprüfen. 



 
Was genau ist ein Smart Home System? Towards Longevity in Smart home Systems kopieren?








\section{Code Listings}

If there is some interesting code you would like to show
in order to ease the understanding of the text,
you can just include it using the \verb+lstlisting+ environment.
Have a look at the source of this page to see how this is included:

\begin{lstlisting}[language=Go]
x := from(42);
\end{lstlisting}

You could also put the code into an external file
and include it in this document using the \verb+lstinputlisting+ command:

\lstinputlisting[language=Go,numbers=left]{listings/example.go}

Be careful not to include large files as it hampers readability.
If there is a short excerpt from a large file you would like to show,
you can also extract an explicit range of lines from it without the need to modify the source file.
This next listing only shows the conditional from the previous code:

\lstinputlisting[language=Go,numbers=left,firstline=2,lastline=5,firstnumber=2]{listings/example.go}

To use more advanced syntax highlighting
have a look at the available options of the \emph{listings} package
or use the \emph{minted} package\footnote{\url{https://texdoc.net/texmf-dist/doc/latex/minted/minted.pdf}},
which has more extensive language support and additional themes.
Both can be configured in the main file.
 % input does not start a new page
\section{Math}

In case you need to include some math,
the \emph{amsmath} package\footnote{\url{https://texdoc.net/texmf-dist/doc/latex/amsmath/amsmath.pdf}} is already included in this document.

To properly display some short formula like \( e^{i \pi} = -1\),
you can use the \verb+\( \)+ inline command.
For larger formulas, the \verb+math+ environment is more appropriate.
If you need to reference the formula multiple times,
e.g. in case it is used in theorems,
you should use the \verb+equation+ environment:

\begin{equation}
	\vec\nabla\times\vec{B}= \mu_0\vec{j}+\mu_0\varepsilon_0\frac{\partial\vec{E}}{\partial t}
	\label{func}
\end{equation}

To reference it as~\ref{func} using the \verb+\ref{}+ command,
remember to use a \verb+\label{}+.

\section{Miscellaneous}

You can use the \verb+\todo{}+ command to put obvious reminders on the side of the document.\todo{like this!}

