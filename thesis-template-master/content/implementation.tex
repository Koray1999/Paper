\chapter{Design und Implementation}\label{ch:eval}

Dieses Kapitel umfasst eine detaillierte Beschreibung des Designs und der Implementation des Algorithmus. Zunächst wird der grobe
Ablauf des Algorithmus beschrieben, um spätere Design- und Implementationsentscheidungen besser nachvollziehen zu können.
Zusätzlich wird ein Generator zum Erstellen von Smart Home Systemen vorgestellt und der mitgegebene Code beschrieben.


\section{Ablauf des Algorithmus}
Zunächst folgt eine grobe Beschreibung aller Schritte des Algorithmus, um so als Leser genauere Entscheidungen im Laufe der Arbeit
besser nachvollziehen zu können. Der Algorithmus basiert auf dem Paper "An Algorithm for Dependency-Preserving Smart
Home Updates" (Zdankin et al, 2020) [1]. Ziel ist es
ist, wie bereits in Kapitel 1.1 erwähnt, Abhängigkeiten zwischen Geräten zu untersuchen und basierend auf diesen Abhängigkeiten
Updatekonfigurationen zu ermitteln, welchen den größtmöglichen Nutzen erbringen. 
\begin{enumerate}
\item Im ersten Schritt wird eine Übersicht über die aktuelle Updatekonfiguration erstellt. Eine Updatekonfiguration ist eine Zusammenfassung
aller Geräte und ihrer aktuellen Firmwareversion. Dadurch erhält man die Information, welche Dienstleistungen die Geräte aktuell anbieten.
Zusätzlich wird eine Liste aller möglichen Updates eines Geräts erstellt. 
\item Der zweite Schritt ist eine Optimierung bei der versucht wird die Anzahl der potenziellen Updates zu verringern. Dazu werden dominierte 
Updates herausgefiltert und im weiteren Verlauf nicht mehr in Betracht gezogen. Wie genau dies geschieht und welche Updates dominiert sind, wird
in Abschnitt 3.2.2 erläutert.
\item Im dritten Schritt wird mit Hilfe der Übersicht aller Geräte und Updates ein Updatekonfigurationsgraph erstellt. Dieser Graph kann in der
Theorie als nichtdeterministischer endlicher Automat realisiert werden, bei dem jeweils ein Zustand eine Updatekonfiguration darstellt. In Abbildung 3.1
ist ein Ausschnitt eines Graphen von drei Geräten mit jeweils zwei Updates zu sehen. Der Start-Zustand zeigt, dass sich Gerät 1 auf Version 0, Gerät 2 auf Version
2 und Gerät 3 auf Version 4 befindet. Die anderen Zustände bilden die anderen Möglichkeiten auf die man die Geräte updaten kann, wodurch andere
Updatekonfigurationen entstehen.

\begin{figure}[h]
\centering
\includegraphics{"images/UCG"}
\caption{Ausschnitt eines Updatekonfigurationgraphs, (Zdankin et al, 2020, An Algorithm for Dependency-Preserving Smart Home Updates) }
\label{fig:Prob1:MEA}
\end{figure}
\item Im vierten Schritt werden alle Updatekonfigurationen dahingehend überprüft, ob sie eine Abhängigkeit verletzen. Ist dies der Fall wird diese
Konfiguration aus dem Updatkonfigurationsgraphen entfernt.
\item Im fünften Schritt werden alle nicht Pareto-optimalen Konfigurationen herausgefiltert. Pareto-optimal wird in Abschnitt 3.2.2 definiert.
\item Im sechsten Schritt werden die Updatekonfigurationen in Subnetzwerke unterteilt. Besteht zum Beispiel keine Abhängigkeit zwischen zwei
Gruppen von Geräten in einem System, kann man diese Gruppen in Subnetzwerke unterteilen. Für diese Gruppen kann man später unabhängig von
anderen Gruppen die passende Updatekonfiguration auswählen.
\item Im letzten Schritt wird jeder Updatekonfiguration ein Rating zugewiesen mit dem Ziel dem Nutzer die Entscheidung für die passende
Updatekonfiguration zu erleichtern.
\end{enumerate}

\section{Generelle Entscheidungen}
Basierend auf den Erkenntnissen aus den Kapiteln 1 und 2 können nun Designentscheidungen bezüglich des Algorithmus getroffen
werden. Ziel ist es den Prototypen eines zukunftsfähigen Tools zu entwickeln, welcher Smart Home Nutzern die Möglichkeit bietet,
Updates vor ihrer Installation zu überprüfen.
Dabei zielt das Design darauf ab unnötige, architekturspezifische Regularien zu vermeiden, um so möglichst simpel
gestaltet werden zu können. Nichtsdestotrotz sollte es möglichst einfach sein, den Algorithmus in verschiedene Smart Home 
Architekturen zu integrieren. Um solch eine einfache Integration zu gewährleisten, wird die Implementation
in Java modular gehalten. Java bietet sich dafür als objektorientierte Programmiersprache an.
Des Weiteren wird der Algorithmus nicht auf der Basis eines realen Smart Home Systems aufgebaut. Dies wäre kontraproduktiv,
da man als Entwickler dazu neigen würde eine systemspezifische Implementation zu entwickeln. Anstatt also ein reales
Smart Home Systems als Grundlage zu nehmen, werden im Folgenden andere Möglichkeiten genauer beleuchtet. 
Eine Möglichkeit ist es, eine Liste mit 
Geräten, aktueller Firmware Version, möglichen Updates und allen dazugehörigen Dienstleistungen zu erstellen. Fügt
man dieser Liste dann noch Abhängigkeiten hinzu, erhält man alle wichtigen Informationen über ein Smart Home System,
die man benötigt. Vorteil einer solchen Liste ist, dass es sehr einfach wäre sie zu erstellen. Für die spätere Evaluation des 
Algorithmus wäre solch eine statische Liste kontraproduktiv, da der Algorithmus somit immer am gleichen 
System getestet werden würde. Dadurch könnten keine validen Aussagen über die Performance des Algorithmus getroffen werden. Aus 
diesem Grund wird ein Generator genutzt, welcher in Abhängigkeit von verschiedenen Parametern Smart Home Systeme kreiert.
So können die Anzahl der Geräte, die Anzahl der möglichen 
Updates pro Gerät, die Anzahl der Dienstleistungen pro Gerät oder auch die Anzahl der Abhängigkeiten zwischen den
Geräten variiert werden. Dies ermöglich den später implementierten Algorithmus an verschiedenen Systemen zu 
evaluieren, um so ein gutes Fazit über Effizienz und Laufzeit treffen zu können. 

\subsection{Generator}
Das Design eines solchen Generators lässt sich in Java folgenderweise gestalten.
Der Generator erzeugt Geräte, welche in Form von Objekten gespeichert werden. Diese Objekte enthalten Informationen über
die Version, die Dienstleistungen und die möglichen Updates des Geräts. Zusätzlich wird noch das Alter der Updates für spätere
Zwecke gespeichert. Idealerweise sollten Updates nach den Spezifikationen des "Semantic Versioning" definiert werden, da es sich
dabei um eine weitverbreitete Art der Versionierung handelt und die Implementation sich damit an aktuelle Standards hält.
Um jedoch im Rahmen dieser Arbeit die Implementierung zu vereinfachen, wird auf diesen Standard verzichtet und Updates werden
im ganzzahligen Bereich von null an hochgezählt. Dementsprechend kommt nach dem Update null das Update eins und so weiter.
\begin{lstlisting}[language=Java]
public class Device {
    int version;
    ArrayList<Integer> services;
    ArrayList<ArrayList<Integer>> updates;
    ArrayList<ArrayList<LocalDate>> updateAge;
}
\end{lstlisting}
Wie bereits in Kapitel 2.1.1 erwähnt unterscheiden sich Smart Home Geräte hinsichtlicher ihrer Definition.
Da ohne klare Definitionen eine Kommunikation zwischen den Geräten nicht
möglich wäre, wird davon ausgegangen, dass alle Geräte und Dienstleistungen unter einem präskriptiven Standard
definiert sind. Die Geräte können also untereinander kommunizieren ohne, dass eine Übersetzung nötig ist, wie es unter einem deskriptiven
Standard der Fall wäre. Rein für die Implementierung des Algorithmus ist es von Vorteil sich an präskriptiven Standards zu orientieren, 
da durch die Homogenität der Geräte und die redundanzfreie Architektur die Komplexität des Algorithmus gemindert werden kann.


Nachdem die Geräte erstellt wurden, müssen Abhängigkeiten zwischen diesen Geräten Abhängigkeiten kreiert werden, um 
ein Smart Home System zu simulieren. Eine Abhängigkeit herrscht zwischen zwei Geräten, wenn ein Gerät Zugriff auf die
Dienstleistungen eines anderen Geräts benötigt. In Abbildung 3.2 sieht man die Abhängigkeit zwischen einer Heizung und einem Thermostat.
Die Heizung greift periodisch auf die Dienstleistung des Thermostats zu, indem es das Thermostat nach der aktuellen Temperatur fragt.
Liegt die Temperatur nun unter einer bestimmten Schwelle würde die Heizung automatisch anspringen und so gewährleisten, dass immer eine
Mindesttemperatur vorhanden ist.

\begin{figure}[h]
\centering
\includegraphics{"images/Heizung"}
\caption{Abhängigkeit zwischen zwei Geräten}
\label{fig:Prob1:MEA}
\end{figure}
\newpage
Die zwei Komponenten Geräte und Abhängigkeiten reichen aus, um ein theoretisches Smart Home System zu bilden, welches für die Anwendung
des Algorithmus völlig ausreicht. Weitere Voraussetzungen sind, dass innerhalb eines solchen Systems keine Updates automatisch
installiert werden und es möglich ist, wie in Kapitel 2.1, beschrieben, sich eine Übersicht über alle Geräte und deren Informationen zu
beschaffen. In realen Systemen müssten diese Informationen in den Metadaten der Geräte
gespeichert werden, um sie leicht abfragen zu können. Da Metadaten bei den meisten
Geräten bereits existieren, sollte es für die Entwickler dieser Geräte kein großes Problem darstellen die Metadaten
um diese Informationen zu ergänzen [1]. Genauere Details, wie zum Beispiel, ob die Kommunikation über eine Cloud läuft, sind nicht
interessant und werden dementsprechend vernachlässigt.

\subsection{Konkretes Design}
Nun steht die Basis für die Implementation des Algorithmus. Im ersten Schritt werden die Updates aller Geräte untersucht. 
Ziel dieses Schrittes ist es, dominierte Updates zu identifizieren und anschließend zu löschen, um so Rechenarbeit einzusparen.
Updates sind dominiert,
wenn ein anderes Update existiert, welches mindestens die gleichen Dienstleistungen anbietet und zusätzlich aktueller ist.

\begin{figure}[h]
\begin{center}
\includegraphics{"images/Versions"}
\caption{Gerät mit 5 Updates (linke Spalte) und den dazugehörigen Dienstleistungen (rechte Spalte) (Zdankin et al, 2020, An Algorithm for Dependency-Preserving Smart Home Updates [1])}
\label{fig:Prob1:MEA}
\end{center}
\end{figure}

\FloatBarrier
In Abbildung 3.3 ist ein Gerät mit fünf verschiedenen Updates/Versionen zu sehen. In diesem Beispiel ist die Version 1.0 von Version 1.1 dominiert,
da Version 1.1 die gleichen Dienstleistungen wie Version 1.0 anbietet und zusätzlich aktueller ist. Das Gleiche gilt für Version 2.0 und 2.1.
Entfernt man die dominierten Updates, bleiben für dieses Gerät drei potenzielle Updates übrig, nämlich Version 1.1, 2.1 und 3.0.
Diese Optimierung an allen Geräten im System durchzuführen spart eine Menge Rechenzeit.
Im nächsten Schritt wird nämlich mit Hilfe des kartesischen Produkts aller restlichen, nicht-dominierten Updates ein Updatekonfigurationsgraph
erstellt. Dieser Graph beinhaltet alle möglichen Updatekonfigurationen, die im System installiert werden könnten. Seien also in einem System zehn
Geräte mit je zwei Updates bedeutet dies, dass es \( 2^{(10)} = 1024\) verschiedene Updatekonfigurationen geben würde. Bei drei Updates pro Gerät
wären es schon \( 3^{(10)} = 59049\) verschiedene Konfigurationen. Dies macht deutlich, dass jedes im ersten Schritt als dominiert erkanntes
Update die Größe des Updatekonfigurationsgraphs erheblich verringern kann. Wie bereits in Abschnitt 3.1 erwähnt, kann der Updatekonfigurationsgraph in der Theorie als
nichtdeterministischer endlicher Automat realisiert werden. In Java wird dieser Automat in Form einer mehrdimensionalen Arrayliste dargestellt.
In Abbildung 3.4 sind zwei Geräte eines Smart Home Systems zu sehen, eine Heizung und ein Thermostat. 
Die Heizung ist momentan auf Version 0 mit den Services  \{1,2\} und besitzt keine Updates.
Das Thermostat ist momentan auf Version 0 mit den Services  \{5,7\} und kann auf Version 1 mit den Services \{6,8\} geupdatet werden.
Der Updatekonfigurationsgraph sieht in Java dann folgendermaßen aus:
\begin{figure}[!h]
\begin{center}
\includegraphics{"images/UCG Java"}
\caption{Updatekonfigurationsgraph in Form einer List in Java}
\label{fig:Prob1:MEA}
\end{center}
\end{figure}
Nun wird der Updatekonfigurationsgraph nach ungültigen Konfigurationen durchsucht. Eine Konfiguration ist ungültig, wenn
Dienstleistungen fehlen, die eigentlich aufgrund einer Abhängigkeit nötig wären. 
Schauen wir uns wieder Abbildung 3.4 an und legen dabei
fest, dass die Heizung Zugriff auf die Dienstleistung mit der Nummer 7 vom Thermostat benötigt. Dies würde bedeuten, dass die Konfiguration
\{\{1,2\},\{5,7\}\} gültig und die Konfiguration \{\{1,2\},\{6,8\}\} ungültig ist. Alle ungültigen Konfigurationen können ignoriert werden, wodurch
die Performance des Algorithmus wieder gesteigert werden kann.
Nachdem bereits dominierte Updates und ungültige Updatekonfigurationen entfernt wurden, werden im nächsten Schritt dominierte Updatekonfigurationen
herausgefiltert, um so wieder die
Performance des Algorithmus zu steigern. Eine Updatekonfiguration ist dominiert, wenn eine weitere Konfiguration existiert, in der alle Updates entweder
gleich alt oder neuer sind. In Abbildung 3.5 ist der Ausschnitt eines Updatekonfigurationsgraphen von drei Geräten mit jeweils drei Updates zu sehen, wobei die Updates
nach ihrer Aktualität von links nach rechts sortiert sind. Bei Gerät1 also ist das Update \{\{1,3\} aktueller als Update \{\{1,2\} und das Update \{\{1,4\} ist aktueller
als Update \{\{1,3\}. Die drei Zustände stehen für je eine Updatekonfiguration, die Pfeile sollen andeuten, dass es möglich ist von einer Updatekonfiguration auf
eine andere upzudaten. Bei genauerer Betrachtung der einzelnen Updatekonfigurationen ist zu erkennen, dass die grüne Konfiguration von der blauen dominiert wird,
von der roten jedoch nicht. In der blauen Konfiguration besitzen nämlich alle Geräte eine aktuellere Version, in der roten Konfiguration hingegen ist die Version des
zweiten Geräts eine Ältere (\{\{6,7\} ist älter als \{\{6,8\}). Des Weiteren ist zu erkennen, dass die rote Konfiguration ebenfalls von der blauen dominiert
wird, denn auch hier besitzen alle Geräte eine aktuellere Version.
\begin{figure}[!h]
\begin{center}
\includegraphics{"images/Dominierte Konfig"}
\caption{Dominiere Updatekonfigurationen}
\label{fig:Prob1:MEA}
\end{center}
\end{figure}

Durch das Filtern dominierter Updatekonfiguration wird die Anzahl potenzieller Konfigurationen signifikant verringert, wodurch es für den Nutzer einfacher wird, sich
für eine passende Konfiguration zu entscheiden. Um diese Entscheidung noch zusätzlich zu vereinfachen, werden im folgenden Schritt Subnetzwerke erstellt, für die
unabhängig von anderen Subnetzwerken eine Updatekonfiguration ausgewählt werden kann. Es ist nicht unüblich, dass in einem Smart Home System Gruppen vorhanden
sind, die unabhängig von anderen Gruppen kommunizieren. Mögliche Gruppen zwischen denen keine Kommunikation notwendig ist, können zum Beispiel Fernseher+Soundbox
und Backofen+Mikrowelle sein. Hier bietet es sich logischerweise an die möglichen Updatekonfigurationen unabhängig voneinander zu betrachten, da so für
den Nutzer ein einfacherer Überblick verschafft werden kann. 
Ein Subnetzwerk besteht also aus allen Geräten, die durch eine Abhängigkeit
in Verbindung stehen. Diese Verbindung muss keine Direkte sein. Seien zum Beispiel zwei Geräte A und B durch eine Abhängigkeit verbunden und gleichzeitig sei Gerät
A zusätzlich mit einem Gerät C aufgrund einer Abhängigkeit verbunden, dann sind die Geräte B und C ebenfalls auf indirektem Weg miteinander verbunden, obwohl
keine direkte Abhängigkeit zwischen ihnen herrscht. Somit sind alle drei Geräte A, B und C Teil eines Subnetzwerkes.
Der letzte Schritt besteht darin Bewertungen für die übrig gebliebenen Updatekonfigurationen zu erstellen. Zdankin schlägt vor zu diesem Zweck den einzelnen Dienstleistungen ein
Wert zuzuschreiben [1]. Unter einem präskriptiven Standard wäre es theoretisch für Hersteller möglich Daten zu sammeln und so zu ermitteln, welche Dienstleistungen am 
häufigsten verwendet werden. Ein Standard könnte dann zusätzlich zur Definition einer Dienstleistung eine Wertung abgeben, je nach dem wie häufig diese Dienstleistung 
im Gegensatz zu anderen verwendet wird. Nutzerstatistiken allein reichen nicht aus um eine valide Wertung abzugeben, da Dienstleistungen existieren, welche trotz
seltener Nutzung extrem wichtig sein können. Zum Beispiel sollte der Alarm eines Feuermelders nicht aufgrund der seltenen Nutzung dieser Dienstleistung, als weniger wichtig
bewertet werden. Daher bedarf es weiterer Metriken, um eine valide Aussage über den Wert einer Dienstleistung treffen zu können [1].

\section{Implementierter Code}
Die mitgegebene Implementation des beschriebenen Codes besteht aus 4 Klassen:
\begin{itemize}
\item Device.java: Klasse zum Erstellen der Geräte als Objekte.
\item Generator.java: Generator zum erstellen verschiedener Smart Home Systeme
\item SmartHome.java: Klasse in der das Smart Home System gespeichert wird und an der der Algorithmus ausgeführt wird.
\item Algorithm.java: Die Implementierung des Algorithmus wie er beschrieben wurde.
\end{itemize}

Die Klasse Device beinhaltet einen Konstruktor zum Erstellen der Geräte. Der Generator nutzt diesen Konstruktor und erstellt mit der
Funktion \textit{createDevice} Geräte in Form von Objekten. Zwischen den Geräten werden anschließend mit der Funktion \textit{createDependencies} Abhängigkeiten erzeugt,
die in einer Arrayliste  gespeichert werden.
Beim Erstellen der Geräte und der Abhängigkeiten wird auf die Klasse Random aus dem "java.util" package zurückgegriffen, um den Geräten
zufällige Dienstleistungen und Abhängigkeiten zuzuweisen. Die Anzahl der Geräte und der Abhängigkeiten hängt von manuell ausgewählten
Parametern in der Klasse SmartHome ab.
\begin{lstlisting}[language=Java]
public class SmartHome extends GeneratorFinal{
... 
    static int nrOfDevices = 10;
    static int nrOfUpdatesPerDevice = 3;
    static int nrOfServicesPerDevice = 10;
    static int nrOfDependencies = nrOfDevices*2;
...
}
\end{lstlisting}
Die Klasse Algorithmus beinhaltet den eigentlichen Code des implementierten Algorithmus.
Im ersten Schritt werden mit der Methode \textit{isDominated} alle dominierten Updates aus der Arrayliste \textit{SmartHomeDevices} entfernt und
die übrig gebliebenen Updates in die Funktion \textit{cartesianProduct} als Parameter eingefügt, um so den Updatekonfigurationsgraphen zu erstellen.
Danach werden mit der Funktion \textit{deleteBreakingConfigurations} alle ungültigen Konfigurationen aus dem \textit{updateConfigurationGraph} entfernt.
Dafür erstellt die Funktion für jede Konfiguration eine Arrayliste mit allen in dieser Konfiguration vorkommenden Dienstleistungen und überprüft
dann, ob alle notwendigen Dienstleistungen in der Liste vorkommen. Wenn dies nicht der Fall ist, wird die Konfiguration markiert und anschließend
aus dem Updatekonfigurationsgraphen entfernt. 
Zusätzlich existiert eine weitere Arrayliste \textit{updateConfigurationGraphDates}, in der die Alter der Updates zu finden sind. Wird also durch die Funktion eine
Konfiguration in der Liste \textit{updateConfigurationGraph} entfernt, dann werden in der Liste \textit{updateConfigurationGraphDates} die dazugehörigen Alter ebenfalls
gelöscht.
Als Nächstes entfernt die Funktion \textit{paretoOptimal} allle dominierten Updatekonfigurationen, wie in Kapitel 3.2.2 beschrieben.
Die Funktionen \textit{findSubnetworks} und \textit{floodFill} identifizieren anschließend Subnetzwerke basierend auf den Abhängigkeiten
zwischen den Geräten. Nachdem die Subnetzwerke erstellt werden, sollte im nächsten Schritt für jedes Subnetzwerk der passende Updatekonfigurationsgraph erstellt
werden. Dies wurde nicht implementiert, doch theoretisch wäre es möglich, indem die nötigen Informationen aus dem originalen Updatekonfigurationsgraphen extrahiert
werden. Sei ein Updatekonfigurationsgraph für die Geräte 1, 2, 3, 4 und 5 gegeben, so könnte für ein Subnetzwerk, das aus den Geräten 1, 2 und 3 besteht, der Graph kopiert und
alle Informationen über die Geräte 4 und 5 gelöscht werden. Dabei würden im Graph des Subnetzwerkes einige identische Konfigurationen entstehen, die dann noch gelöscht werden
müssten.
Ein Subnetzwerk besteht dann nur noch aus Geräten zwischen denen direkt oder indirekt eine Abhängigkeit besteht.
Im letzten Schritt erstellt die Funktion \textit{rating} für jede Dienstleistung eine Wertung. Die Werte der einzelnen Dienstleistungen werden dann pro
Subnetzwerk summiert, um sie vergleichen zu können. Ein Nutzer könnte sich anschließend für jedes Subnetzwerk für eine Updatekonfiguration entscheiden.



































