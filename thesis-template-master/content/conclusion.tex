\chapter{Fazit}
Der Fokus dieser Arbeit liegt auf der Implementation und Evaluation eines Algorithmus für Abhängigkeitsbewahrende
Updatekonfigurationen in Smart Home Systemen. Dieser Algorithmus ist in der Theorie ein Werkzeug, welches vielen Smart
Home Nutzern mehr Kontrolle und Sicherheit über ihr System bieten würde.
Die Ergebnisse der Evaluation haben gezeigt, dass der im Paper "An Algorithm for Dependency-Preserving Smart Home Updates" (Zdanking et al. )
beschriebene Algorithmus erfolgreich in einer Implementation umgesetzt wurde. Die Messungen der Laufzeit und des Speicherbedarfs haben
gezeigt, dass der Algorithmus theoretisch an aktuellen Systemen anwendbar wäre. In der Realität sind die in Kapitel 2 und 3 erwähnten Voraussetzungen
nicht alle erfüllt, weswegen es sich bei dem Algorithmus lediglich um ein theoretische Implementation handelt.
Des Weiteren muss erwähnt werden, dass es bei steigender Größe von Smart Home Systeme
zu 
Problemen bezüglich der Laufzeit kommen. Dementsprechend sind weitere Optimierungen bezüglich des Designs
und der Implementation nötig, um die Zukunftsfähigkeit des Algorithmus gewährleisten zu können.
Außerdem muss angemerkt werden, dass die Ergebnisse der Evaluation unter
Vorbehalt zu betrachten sind, da es sich bei den Messungen lediglich um theoretische Ergebnisse handelt, die sehr von der
Implementation des Generators abhängig sind. Dementsprechend wäre es für zukünftige Arbeiten von Vorteil empirische Daten
zu sammeln, um den Generator so konfigurieren zu können, so dass die generierten Systeme möglichst realistisch sind. Um noch
genauere Ergebnisse zu erzielen, wäre es nötig den Algorithmus an realen Smart Home Systemen zu testen. Dafür muss vonseiten
der Hersteller ein Standard eingeführt werden, um einen einheitlichen Zugriff auf die Informationen eines Smart Home Systemen zu
gewährleisten. Ein Vorschlag diesbezüglich ist in Kapitel 2.1.2 zu finden. Ohne einen solchen Standard wäre es nicht möglich den Algorithmus
plattformunabhängig zu realisieren.
Insgesamt lässt sich sagen, dass es sich bei dieser Arbeit, trotz der Implementierung, um ein sehr theoretisches Ergebnis handelt. Viele Aspekte,
wie zum Beispiel, dass ein Smart home System unter einem präskriptiven Standard definiert ist, wurden vorausgesetzt, sind in der Realität jedoch
nicht gegeben. Dementsprechend
müssen die verschiedenen Anbieter von Smart Home System zusammenarbeiten und gemeinsam Standards definieren, um einen Algorithmus
wie diesen in der Realität umsetzen zu können.




\begin{thebibliography}{9}

%1
\bibitem{texbook}
Peter Zdankin, Matthias Schaffeld, Marian Waltereit, Oskar Carl, Torben Weis, "An Algorithm for Dependency-Preserving Smart Home Updates" (2021).
\url{https://ieeexplore.ieee.org/document/9431040}

%2
\bibitem{texbook}
Global Smart Home Market Size By Technologies (Cellular Network Technologies, Protocols And Standards), By Product
(Lighting Control, Security And Access Control, HVAC Control), By Geographic Scope And Forecast
\url{https://www.verifiedmarketresearch.com/product/global-smart-home-market-size-and-forecast-to-2025/} (Visited on 23.08.2021)

%3
\bibitem{textbook}
\url{https://de.wikipedia.org/wiki/Smart_Home} (visited on 18.08.2021)

%4
\bibitem{texbook}
Expertenbeitrag von Amanuel Dag, Country Director bei CONTEXT
\url{https://www.homeandsmart.de/smart-home-status-quo-2019} (visited on 18.08.2021)

%5
\bibitem{textbook}
\url{https://policyadvice.net/insurance/insights/smart-home-statistics/)} (visited on 18.08.2021)

%6
\bibitem{textbook}
Matthias Kersken, Herbert Sinnesbichler, Hans Erhorn, "Analyse der Einsparpotenziale durch Smarthome- und intelligente Heizungsregelungen" (Oktober 2018)
\url{https://www.ibp.fraunhofer.de/content/dam/ibp/ibp-neu/de/dokumente/sonderdrucke/bauphysik-gertis/6-einsparpotenziale-intelligente-heizungsregelung.pdf} (visited on 03.09.2021)

%7
\bibitem{textbook}
Peter Zdankin, Matthias Schaffeld, Marian Waltereit, Oskar Carl, Torben Weis, "Towards Longevity of Smart Home Systems" (2020),
2020 IEEE International Conference on Pervasive Computing and Communications Workshops (PerCom Workshops)
\url{https://ieeexplore.ieee.org/abstract/document/9156165}

%8
\bibitem{textbook}
Peter Zdankin, Matthias Schaffeld, Marian Waltereit, Oskar Carl, Torben Weis, "Requirements and Mechanisms for Smart Home Updates" (october 2020)
\url{https://www.researchgate.net/publication/344441146_Requirements_and_Mechanisms_for_Smart_Home_Updates}

%9
\bibitem{textbook}
\url{https://mitosystems.com/software-evolution/} (visited on 03.09.2021)

%10
\bibitem{textbook}
P. D. R. H. Reussner. (2012) SPP 1593: Design
for Future - Managed Software Evolution.
\url{https://gepris.dfg.de/gepris/projekt/198572722?context=projekt&task=showDetail\&id=198572722} (visited on 03.09.2021)

%11
\bibitem{textbook}
\url{https://en.wikipedia.org/wiki/Dependency_hell} (visited on 03.09.2021)

%12
\bibitem{textbook}
\url{https://semver.org/} (visited on 03.09.2021)

%13
\bibitem{textbook}
\url{https://arstechnica.com/gadgets/2018/12/logitech-firmware-update-breaks-locally-
controlled-harmony-hub-systems/} (visited on 28.08.2021)

%\bibitem{textbook}
%https://techjury.net/blog/how-many-iot-devices-are-there

%14
\bibitem{textbook}
\url{https://venturebeat.com/2018/02/28/apples-ios-update-frequency-has-increased-51-
under-cooks-management} (visited on 06.09.2021)

%15
\bibitem{textbook}
\url{https://www.businessinsider.de/insider-picks/technik/die-meistverkauften-smartphones-2021} (visited on 12.09.2021)

%16
\bibitem{textbook}
\url{https://de.wikipedia.org/wiki/IPhone\_12} (visited on 12.09.2021)

%17
\bibitem{textbook}
\url{https://www.cloudplan.net/blogdetail/Was-ist-eine-Cloud-und-wie-funktioniert-sie} (visited on 15.09.2021)

%18
\bibitem{textbook}
Jakob Nielsen, "Usability Engineering" (1993), Chapter 5
\url{https://www.nngroup.com/articles/response-times-3-important-limits/} (visited on 16.09.2021)

%19
\bibitem{textbook}
\url{https://www.statista.com/statistics/1107206/average-number-of-connected-devices-us-house/} (visited on 16.09.2021)

\end{thebibliography}
