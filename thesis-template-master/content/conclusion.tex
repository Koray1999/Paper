\chapter{Conclusion}

The conclusion quickly summarizes the results of the paper
in relation to the previously defined goals and hypotheses.
It usually also includes some information on next steps and further research
that might be required or possible on the presented subject.



\section{Bibliograhpy}

\begin{thebibliography}{9}

\bibitem{texbook}
Peter Zdankin, Matthias Schaffeld, Marian Waltereit, Oskar Carl, Torben Weis, "An Algorithm for Dependency-Preserving Smart"
Home Updates"

\bibitem{texbook}
https://www.verifiedmarketresearch.com/product/global-smart-home-market-size-and-forecast-to-2025/

\bibitem{texbook}
https://www.homeandsmart.de/



\bibitem{textbook}
(https://policyadvice.net/insurance/insights/smart-home-statistics/)

\bibitem{textbook}
(https://www.ibp.fraunhofer.de/content/dam/ibp/ibp-neu/de/dokumente/sonderdrucke/bauphysik-gertis/6-einsparpotenziale-intelligente-heizungsregelung.pdf

\bibitem{textbook}
https://mitosystems.com/software-evolution/

\end{thebibliography}

\section{Code Listings}

If there is some interesting code you would like to show
in order to ease the understanding of the text,
you can just include it using the \verb+lstlisting+ environment.
Have a look at the source of this page to see how this is included:

\begin{lstlisting}[language=Go]
x := from(42);
\end{lstlisting}

You could also put the code into an external file
and include it in this document using the \verb+lstinputlisting+ command:

\lstinputlisting[language=Go,numbers=left]{listings/example.go}

Be careful not to include large files as it hampers readability.
If there is a short excerpt from a large file you would like to show,
you can also extract an explicit range of lines from it without the need to modify the source file.
This next listing only shows the conditional from the previous code:

\lstinputlisting[language=Go,numbers=left,firstline=2,lastline=5,firstnumber=2]{listings/example.go}

To use more advanced syntax highlighting
have a look at the available options of the \emph{listings} package
or use the \emph{minted} package\footnote{\url{https://texdoc.net/texmf-dist/doc/latex/minted/minted.pdf}},
which has more extensive language support and additional themes.
Both can be configured in the main file.
 % input does not start a new page
\section{Math}

In case you need to include some math,
the \emph{amsmath} package\footnote{\url{https://texdoc.net/texmf-dist/doc/latex/amsmath/amsmath.pdf}} is already included in this document.

To properly display some short formula like \( e^{i \pi} = -1\),
you can use the \verb+\( \)+ inline command.
For larger formulas, the \verb+math+ environment is more appropriate.
If you need to reference the formula multiple times,
e.g. in case it is used in theorems,
you should use the \verb+equation+ environment:

\begin{equation}
	\vec\nabla\times\vec{B}= \mu_0\vec{j}+\mu_0\varepsilon_0\frac{\partial\vec{E}}{\partial t}
	\label{func}
\end{equation}

To reference it as~\ref{func} using the \verb+\ref{}+ command,
remember to use a \verb+\label{}+.

\section{Miscellaneous}

You can use the \verb+\todo{}+ command to put obvious reminders on the side of the document.\todo{like this!}
