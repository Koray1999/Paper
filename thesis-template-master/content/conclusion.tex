\chapter{Conclusion}

The conclusion quickly summarizes the results of the paper
in relation to the previously defined goals and hypotheses.
It usually also includes some information on next steps and further research
that might be required or possible on the presented subject.



\section{Bibliograhpy}

\begin{thebibliography}{9}

\bibitem{texbook}
Peter Zdankin, Matthias Schaffeld, Marian Waltereit, Oskar Carl, Torben Weis, "An Algorithm for Dependency-Preserving Smart"
Home Updates"

\bibitem{texbook}
https://www.verifiedmarketresearch.com/product/global-smart-home-market-size-and-forecast-to-2025/

\bibitem{texbook}
https://www.homeandsmart.de/



\bibitem{textbook}
(https://policyadvice.net/insurance/insights/smart-home-statistics/)

\bibitem{textbook}
(https://www.ibp.fraunhofer.de/content/dam/ibp/ibp-neu/de/dokumente/sonderdrucke/bauphysik-gertis/6-einsparpotenziale-intelligente-heizungsregelung.pdf

\bibitem{textbook}
https://mitosystems.com/software-evolution/

\end{thebibliography}

\input{content/related_work/listings} % input does not start a new page
\input{content/related_work/math}
\input{content/related_work/misc}