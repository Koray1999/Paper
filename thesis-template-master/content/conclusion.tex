\chapter{Conclusion}
Der Fokus dieser Arbeit liegt auf der Implementation und Evaluation eines Algorithmus für Abhängigkeitsbewahrende
Updatekonfigurationen in Smart Home Systemen. Dieser Algorithmus ist in der Theorie ein Werkzeug, das vielen Smart
Home Nutzern mehr Kontrolle und Sicherheit über ihr System bieten würde.
Die Ergebnisse der Evaluation haben gezeigt, dass das Design des Algorithmus erfolgreich in einer Implementation
umgesetzt wurde. Steigt die Größe der Smart Home Systeme jedoch an, kann es zu 
Problemen bezüglich der Laufzeit und des Speicherbedarfs kommen. Dementsprechend sind weitere Optimierungen bezüglich des Designs
und der Implementation nötig, um die zukunftsfähigkeit des Algorithmus gewährleisten zu können.
Insgesamt liegt das Ziel dieser Arbeit darin die Langlebigkeit von Smart Home Systemen
gewährleisten zu können. Es gibt viele weitere Aspekte, die einen Einfluss auf die Langlebigkeit eines Smart Home Systems
haben. Einige dieser Aspekte wurden in dieser Arbeit angerissen, jedoch nicht genauer thematisiert. In Folgearbeiten könnten
diese Aspekt genauer untersucht werden. Des Weiteren muss angemerkt werden, dass die Ergebnisse der Evaluation unter
Vorbehalt zu betrachten sind, da es sich bei den Messungen lediglich um theoretische Ergebnisse handelt, die sehr von der
Implementation des Generators abhängig sind. Dementsprechend wäre es für zukünftige Arbeiten von Vorteil empirische Daten
zu sammeln, um den Generator so konfigurieren zu können, so dass die generierten Systeme möglichst realistisch sind. Um noch
genauere Ergebnisse zu erzielen wäre es nötig den Algorithmus an realen Smart Home Systemen zu testen. Dafür muss von Seiten
der Hersteller ein Standard eingeführt werden, um einen einheitlichen Zugriff auf die Informationen eines Smart Home Systemen zu
haben. Ein Vorschlag diesbezüglich ist in Kapitel 2.1.2 zu finden. Ohne einen solchen Standard wäre es nicht möglich den Algorithmus
Plattformunabhängig zu realisieren. Außerdem wurden viele Detailaspekte in dieser Arbeit außer Acht gelassen. So wurde nicht
festgelegt, 




\begin{thebibliography}{9}

\bibitem{texbook}
Peter Zdankin, Matthias Schaffeld, Marian Waltereit, Oskar Carl, Torben Weis, "An Algorithm for Dependency-Preserving Smart Home Updates" (2021).
\url{https://ieeexplore.ieee.org/document/9431040}

\bibitem{texbook}
Global Smart Home Market Size By Technologies (Cellular Network Technologies, Protocols And Standards), By Product
(Lighting Control, Security And Access Control, HVAC Control), By Geographic Scope And Forecast
\url{https://www.verifiedmarketresearch.com/product/global-smart-home-market-size-and-forecast-to-2025/} (Visited on 23.08.2021)

\bibitem{textbook}
\url{https://de.wikipedia.org/wiki/Smart_Home} (visited on 18.08.2021)

\bibitem{texbook}
Expertenbeitrag von Amanuel Dag, Country Director bei CONTEXT
\url{https://www.homeandsmart.de/smart-home-status-quo-2019} (visited on 18.08.2021)

\bibitem{textbook}
\url{https://policyadvice.net/insurance/insights/smart-home-statistics/)} (visited on 18.08.2021)

\bibitem{textbook}
Matthias Kersken, Herbert Sinnesbichler, Hans Erhorn, "Analyse der Einsparpotenziale durch Smarthome- und intelligente Heizungsregelungen" (Oktober 2018)
\url{https://www.ibp.fraunhofer.de/content/dam/ibp/ibp-neu/de/dokumente/sonderdrucke/bauphysik-gertis/6-einsparpotenziale-intelligente-heizungsregelung.pdf} (visited on 03.09.2021)

\bibitem{textbook}
Peter Zdankin, Matthias Schaffeld, Marian Waltereit, Oskar Carl, Torben Weis, "An Algorithm for Dependency-Preserving Smart Home Updates" (2020),
2020 IEEE International Conference on Pervasive Computing and Communications Workshops (PerCom Workshops)
\url{https://ieeexplore.ieee.org/abstract/document/9156165}

\bibitem{textbook}
Peter Zdankin, Matthias Schaffeld, Marian Waltereit, Oskar Carl, Torben Weis, "Requirements and Mechanisms for Smart Home Updates" (october 2020)
\url{https://www.researchgate.net/publication/344441146_Requirements_and_Mechanisms_for_Smart_Home_Updates}

\bibitem{textbook}
\url{https://mitosystems.com/software-evolution/} (visited on 03.09.2021)

\bibitem{textbook}
F. Doesburg, F. Cnossen, W. Dieperink, W. Bult, A. M. de Smet, D. J.
Touw, and M. W. Nijsten, “Improved usability of a multi-infusion
setup using a centralized control interface: A task-based usability test,”
PLOS ONE, vol. 12, no. 8, pp. 1–10, 08 2017. [Online]. Available:
\url{https://doi.org/10.1371/journal.pone.0183104} (visited on 03.09.2021)

\bibitem{textbook}
\url{https://en.wikipedia.org/wiki/Dependency_hell} (visited on 03.09.2021)

\bibitem{textbook}
\url{https://semver.org/} (visited on 03.09.2021)

\bibitem{textbook}
\url{https://arstechnica.com/gadgets/2018/12/logitech-firmware-update-breaks-locally-
controlled-harmony-hub-systems/} (visited on 28.08.2021)

%\bibitem{textbook}
%https://techjury.net/blog/how-many-iot-devices-are-there

\bibitem{textbook}
\url{https://venturebeat.com/2018/02/28/apples-ios-update-frequency-has-increased-51-
under-cooks-management} (visited on 06.09.2021)

\bibitem{textbook}
\url{https://www.businessinsider.de/insider-picks/technik/die-meistverkauften-smartphones-2021} (visited on 12.09.2021)

\bibitem{textbook}
\url{https://de.wikipedia.org/wiki/IPhone\_12} (visited on 12.09.2021)



\end{thebibliography}
